\documentclass[12pt]{article}
\usepackage[margin=6mm]{geometry}
\usepackage{amssymb, microtype, mathtools, booktabs}
\usepackage[table]{xcolor}

\begin{document}

\paragraph{Description} The properties of limits are intuitive but important for simplifying problems.

\paragraph{Setup}
\begingroup \large
\begin{align*}
  \lim_{x \to c} f(x) = L && \lim_{x \to c} g(x) = M
\end{align*}
\endgroup

\paragraph{Limits Properties}
\begin{center}
\LARGE \def\arraystretch{3} % stretch row size
\rowcolors{1}{white}{gray!10}
\begin{tabular}{cccc}
  Sum Property & $\lim\limits_{x \to c} (f(x) + g(x))$ & = & $L + M$ \\
  Difference Property & $\lim\limits_{x \to c} (f(x) \times g(x))$ &
  = & $L -M$ \\
  Constant Multiple Property & $\lim\limits_{x \to c} (k \times f(x))$ &
  = & $k \times L$ \\
  Quotient Property & \LARGE $\frac{\lim_{x \to c} (f(x))}
  {\lim_{x \to c} (g(x))}$ & = & $L \div K$ \\
  Exponent Property & $\lim\limits_{x \to c} {(f(x))}^{\frac{r}{s}}$ &
  = & $L^{\frac{r}{s}}$
\end{tabular}
\end{center}

\end{document}
